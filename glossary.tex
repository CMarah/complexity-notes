\documentclass[12pt]{article}
\usepackage[margin=15mm]{geometry}
\usepackage[english]{babel}
\usepackage[latin1]{inputenc}
\usepackage{amsmath}
\usepackage{amsthm}
\usepackage{amssymb}
\usepackage{amsfonts}
\usepackage{bm}

\title{Mathematics and Computation Glossary}
\author{Carlos Mar�a Rodr�guez}
\date{\today}

\newtheoremstyle{customdef}
  {5mm}%   Space above
  {0mm}%   Space below
  {\slshape}%  Body font
  {}%          Indent amount (empty = no indent, \parindent = para indent)
  {\bfseries}% Thm head font
  {.}%         Punctuation after thm head
  {0.5em}%     Space after thm head: " " = normal interword space;
  {}%          Thm head spec (can be left empty, meaning `normal')

\theoremstyle{customdef}
\newtheorem{defn}{Def.}[section]
\theoremstyle{plain}
\newtheorem{thm}{Theorem}[section]

\begin{document}
  \maketitle

  \section{Basics}

  \begin{defn}$\bm{I}$ denotes the set of all binary sequences of all lengths. $\bm{I_n}$ denotes
  the sequence of all binary sequences in $\bm{I}$, this is, $\bm{I_n} = \{0,1\}^n$.\end{defn}

  \begin{defn}(The class $\bm{\mathcal{P}}$). \\
    A function $f: \bm{I} \rightarrow \bm{I}$ is in the
    class $\mathcal{P}$ if there is an algorithm computing $f$ and positive constants, $A$,
    $c$, such that $\forall n \in \mathbb{N}, \forall x \in I_n$, the algorithm computes
    $f(x)$ in at most $An^c$ steps.
  \end{defn}

\begin{defn}(The class $\bm{\mathcal{NP}}$).\\
  The set $\mathcal{C} \in \bm{I}$ is in the class
    $\mathcal{NP}$ if there is a function $V_C \in \mathcal{P}$ and a constant $k \in
    \mathbb{R}$ such that:
    \begin{itemize}
      \item If $x \in \mathcal{C}$, then $\exists y \in \mathbb{R}$ with $|y| \leq
        k|x|^k$ and $V_C(x,y) = 1$.
      \item If $x \notin \mathcal{P}$, then $\forall y$ we have $V_C(x,y) = 0$.
    \end{itemize}
  \end{defn}
  The function $V_C$ is called the \emph{verification algorithm}, and the sequence $y$
  for which $V_C(x,y) = 1$ is called the \emph{witness}.


\begin{defn}(The class $\bm{co\mathcal{NP}}$).

  A set $\mathcal{C} \in \bm{I}$ is
    in the class $co\mathcal{NP}$ iff its complement
    $\bar{\mathcal{C}} = \bm{I} \setminus \mathcal{C}$ is in
$\mathcal{P}$.\end{defn}

  \begin{defn}(Efficient reductions).

    Let $C,D \subset \bm{I}$ be two classification
problems. $f: \bm{I} \rightarrow \bm{I}$ is an efficient reduction from $C$ to $D$ if
$f \in \mathcal{P}$ and $\forall x \in \bm{I}$ we have $x \in C \iff x \in D$.\end{defn}
We write $C \leq D$ if such a reduction exists.

\begin{defn}(Hardness and completeness).

  A problem $D$ is called $\mathcal{C}$-hard if $\forall C \in \mathcal{C}$,
we have $C \leq D$. If we further have that $D \in \mathcal{C}$, then $D$ is called
$\mathcal{C}$-complete.\end{defn}

\begin{defn}(The $SAT$ problem).\\
  Given a logical expression over Boolean variables (can take values in $\{0, 1\}$
  with connectives $\wedge, \vee, \neg$), is it satisfiable?
  This is, is there a boolean assignment of the variables trough which the expression
  evalutates to $1$? The set of all such expressions is denoted by $SAT$.
\end{defn}

\begin{thm}$SAT$ is $\mathcal{NP}$-complete.\end{thm}

\begin{defn}(The $2DIO$ problem).\\
  Given a Diophantine equations of the form $Ax^2 + By + C = 0; A,B,C \in \mathbb{Z}$,
  is it solvable with positive integers?
\end{defn}

\begin{thm}$2DIO$ is $\mathcal{NP}$-complete.\end{thm}

\begin{defn}(The $3COL$ problem).\\
  Given a planar map, can you color it using only 3 different colors?
\end{defn}

\begin{thm}$3COL$ is $\mathcal{NP}$-complete.\end{thm}

\begin{defn}(The $subset-sum$ problem).\\
  Given a sequence $a_1, a_2, ..., a_n \in \mathbb{Z}$ and $b$, is there a subset $J$ such
  that $\sum_{i \in J} a_i = b$?
\end{defn}

\begin{thm}$Subset-sum$ is $\mathcal{NP}$-complete.\end{thm}
\end{document}
